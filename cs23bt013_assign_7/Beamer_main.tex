\documentclass{beamer}
\usepackage[utf8]{inputenc}
\usetheme{Madrid}

\setbeamertemplate{footline}{}
\setbeamertemplate{itemize item}{\rule[0.3ex]{1ex}{1ex}}

\newtheorem{Router}{Router}
\newtheorem{Firewall}{Firewall}
\newtheorem{Firewalls}{Firewalls}
\newtheorem{Switch}{Switch}
\newtheorem{Encryption}{Encryption}
\newtheorem{VPN}{VPN}

\usepackage{graphicx}
\usepackage{caption}
\title{Basics of Computer Networking}
\author{Jaswika Maryada \\
          CS23BT013
          }
\begin{document}
\begin{frame}
\titlepage
\end{frame}
\begin{frame}{What is Computer Networking?}
 Computer Networking is the practice of interconnecting computers
 to share resources and exchange data. Networks can be as simple
 as connecting two computers or as complex as the global Internet.
\end{frame}
\begin{frame}{Types of Networks}
 Computer networks can be classified into different types based on
 their size and purpose:
    \begin{itemize}
        \item \textbf{LAN} -Local Area Network, typically within a building or
        
 campus.
        \item \textbf{VAN}- Wide Area Network, spanning large geographical
        
 areas.
        \item \textbf{MAN}- Metropolitan Area Network, larger than LAN
        
        but smaller than WAN.
 \end{itemize}
\end{frame}
\begin{frame}{Network Topologies}
  Network topology refers to the layout of how devices are connected
 in a network:
  \begin{itemize}
        \item \textbf{Bus Topology} -All devices share a single communication line.
        \item \textbf{Star Topology}-  Devices are connected to a central hub.
        \item \textbf{Ring Topology}- Devices form a closed loop.
 \item \textbf{Mesh Topology}- Devices are interconnected with many
 redundant connections.

 \end{itemize}
\end{frame}
    \begin{frame}{Networking Layers}
    The OSI model is divided into seven layers, which allows for
 abstraction and modularity in networking.
 \begin{itemize}
     \item Physical Layer\pause
     \item  Data Link Layer \pause
     \item   Network Layer \pause
     \item  Transport Layer \pause
     \item  Session Layer \pause
     \item  Presentation Layer\pause
     \item  Application Layer
 \end{itemize}
\end{frame}
\begin{frame}{Network Protocols Comparison}
 Here’s  a comparison of some common 
 networking protocols:
 \vspace{0.2cm}
 \centering
     \vspace{0.5cm}
 \begin{tabular}{|c|c|c|c|}
     \hline
     \textbf{Protocol} & \textbf{Layer} & \textbf{Use Case} & \textbf{Port Number} \\ \hline
      HTTP & Application & Web Traffic & 80 \\
      \hline
       FTP & Application & File Transfer & 21 \\
      \hline
       TCP & Transport &  Reliable Data Transfer &  N/A \\
      \hline
     Ethernet & Data Link &    LANCommunication &  N/A \\
      \hline
      
 \end{tabular}
\end{frame}
\begin{frame}{Data Transmission Formula}
     The data transmission rate formula in networking is:
     \[
    \text{Data Rate} = \text{Bandwidth} \times \log_2{\left( 1 + \text{Signal-to-Noise Ratio} \right)}
    \]
 where:
 \begin{itemize}
     \item  Bandwidth is the maximum frequency range (in Hz).
     \item  Signal-to-Noise Ratio(SNR)  is a measure of signal strength
 relative to background noise.
\end{itemize}
\end{frame}
\begin{frame}{Key Networking Components}
   \begin{block}{Router}
        A device that forwards data packets between networks
   \end{block}
    \begin{block}{Switch}
    A device that connects devices within a LAN, managing data
 transmission between them.
    \end{block}
    \begin{block}{Firewall}
    A security system that monitors and controls network traffic based
 on security rules.
    \end{block}
\end{frame}
\begin{frame}{IP Addressing}
    Each device in a network has a unique IP address for identification. IP addresses are of two main types:
    
   {IPv4}: \texttt{192.168.1.1} \hspace{0.2cm} {IPv6}: \texttt{2001:0db8:85a3:0000:0000:8a2e:0370:7334}

    \\[1.5ex] 
    IPv4 is widely used, but IPv6 addresses are becoming more common to accommodate more devices.
\end{frame}
\begin{frame}{OSI Model}
    The OSI model defines a layered framework for how data is transmitted over networks:
     \begin{itemize}
        \item \textbf{Layer 1:Physical} -Hardware transmission (cables, signals).
        \item \textbf{ Layer 2: Data Link}-  Error detection and frame control.
        \item \textbf{ Layer 3: Network}- Routing of packets (e.g., IP).
 \item \textbf{Layer 4: Transport}-  End-to-end communication (e.g.,TCP).
\item \textbf{Layer 5-7}-  Session, Presentation, and Application layers.

 \end{itemize}
\end{frame}
\begin{frame}{Sample Network Diagram}
     \begin{center}
        \includegraphics[width=0.6\textwidth]{network}
       

    \end{center}
    \centering
         Example layout of a small network with routers, switches, and
     connected devices.
\end{frame}
\begin{frame}{Common Network Protocols}
 \begin{itemize}
        \item \textbf{ HTTP/HTTPS} - For web browsing and secure data
 transmission
        \item \textbf{FTP/SFTP}- For file transfers.
        \item \textbf{SMTP}-  For sending emails.
 \item \textbf{DNS}-  Domain Name System, for resolving domain names to
 IP addresses.

 \end{itemize}

\end{frame}
\begin{frame}{Network Security Basics}
    \begin{block}{Encryption}
Protects data by encoding it , making it accessible only to those
 with the correct key
    \end{block}
    \begin{block}{Firewalls}
        Monitors and controls incoming and out going network traffic based
 on security policies.
    \end{block}
    \begin{block}{VPN}
        A Virtual Private Network creates a secure connection over a
 public network
    \end{block}
\end{frame}
\begin{frame}{Conclusion}
     Networking enables connectivity across devices and allows sharing
 of resources. Understanding basic network types, topologies, and
 security measures is essential for creating and maintaining effective
 networks.

\end{frame}
\begin{frame}{Further Reading}
    \begin{itemize}
        \item \textbf{ Books} -  "Computer Networking: A Top-Down Approach" by
 Kurose and Ross.
        \item \textbf{Websites:}- Cisco Networking Academy, Network World.
        \item \textbf{Courses: }-   “Introduction to Computer Networking” on
 coursera

 \end{itemize}
\end{frame}
\end{document}
