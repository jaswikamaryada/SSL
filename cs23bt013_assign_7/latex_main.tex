\documentclass[10pt,a4]{article}
\usepackage{amsmath}
\usepackage{graphicx}
\usepackage{algorithm}
\usepackage{algpseudocode}
\usepackage{xcolor}
\usepackage{natbib}
\usepackage{float}

\title{SSL Latex Assignment}
\author{Jaswika Maryada \\ \\ CS23BT013 \\ \\ Department of Computer Science, IIT Dharwad}

\begin{document}
\maketitle
\newpage
\tableofcontents
\newpage
\section{Introduction}
 This document serves as a comprehensive guide to various mathematical and
 technical concepts, presented in a structured and easy-to-follow format. The
 guide is organized into the following sections:
\begin{itemize}
\item [ ] \textbf{Section 2: Mathematics}
explores fundamental mathematical concepts, be-
ginning with trigonometric equations and identities, and moving through topics
such as matrices, radicals, integration, summation, differentiation, and the use of nested brackets. This section aims to reinforce key principles and provide useful equations and formulas for each area.

\item [ ] \textbf{Section 3: Tables and Figures}
demonstrates how to include tables and figures in a LaTeX document, showcasing ways to organize data and visualize information effectively.

\item [ ] \textbf{Section 4: Numbering} explains various types of lists and numbering conventions, including unordered (bullet) lists, ordered (numbered) lists, Roman numerals, alphabetic lists, and nested lists. The section provides examples to help readers apply these styles in their own documents

\item [ ] \textbf{Section 5: Pseudocode} introduces techniques for writing pseudocode in La-TeX, allowing readers to structure algorithms and complex procedures in a clear,readable format.

\item [ ] \textbf{Section 6: Coloring} covers the use of color in LaTeX, showing how to apply different colors to text, backgrounds, and other elements to enhance readability and aesthetic appeal.

\item [ ] \textbf{Section 7: Citing the Papers} explains the methods for citing academic papers and references in LaTeX, which is essential for creating well-documented technical or research papers.\\

This document includes both a list of figures and a list of tables, helping the
reader quickly locate visual elements. Each section is carefully structured to en-sure clarity and utility, making this guide a valuable resource for anyone seeking to enhance their skills in LaTeX and mathematical documentation.
\end{itemize}

\newpage
\section{Mathematics}
\subsection{Writing Trigonometric equations}
There are six trigonometric functions those are
\begin{itemize}
\item $\sin x$
\item $\cos x$
\item $\tan x$
\item $\cot x$
\item $\sec x$
\item $\csc x$
\end{itemize}

\subsubsection{Some trigonometric identities}
\begin{itemize}
\item $ \sin^2 x + \cos^2 x = 1$
\item  $ \sin^2 \theta + \cos^2 theta = 1$
\item $ 1 + \tan^2 x =\sec^2 x$
\item $ 1+ \cot^2 x = \csc^2 x$
\item $ \tan (2x) = \frac{2\tan x}{1-\tan^2 x}$
\end{itemize}

\subsection{Matrices}
\begin{itemize}
\item Row Vector: A 1xN matrix\\
\(
\left[
\begin{matrix}
a & b & c 
\end{matrix}
\right]
\)

\item Column Vector: A Nx1 matrix\\
\(
\left[
\begin{matrix}
a \\
b\\
c
\end{matrix}
\right] 
\)

\item Square Matrix: NxN matrix\\
\(
\left[
\begin{matrix}
a & b & c \\
d & e & f \\
g & h & i
\end{matrix}
\right]
\)
\end{itemize}
\newpage

\subsection{Radicals}
\begin{itemize}
\item Basic Square Root:\\
$ \sqrt{a} $
\item Square Root with an Expression Inside:\\
$ \sqrt{x^2 + y^2} $
\item cube Root:\\
$ \sqrt[3]{a}$
\item nth Root:\\
$ \sqrt[n]{a}$
\end{itemize}
\subsection{Integration}
In this section, we calculatethe integral of a 
function.Theintegral of f(x) = $ x^2$
from 0 to 1 can be written as:\\
\[\int_0^1 x^2 dx\]
This evaluates to:\\
\[\frac{1}{3}x^3 |_0^1 = \frac{1}{3}\]

\subsection{Summation}
 In this section, we calculate the sum of the first n natural numbers. The formula
 for this sum can be written as:\\
 \[S=\sum_{i=1}^n i\]
 This evaluates to:\\
 \[S=\frac{n(n+1)}{2}\]
 For example, when n = 5, the sum is:\\
 \[S=\sum_{i=1}^5i=1+2+3+4+5=15\]
 
 \newpage
 \subsection{differentiation}
  In this section, we discuss the differentiation of a function. The derivative of a
 function f(x) with respect to x is written as:\\
 \[\frac{d}{dx}f(x)\]
 For example,if f(x)=$ x^3 $,its derivative is:
 \[\frac{d}{dx}x^3=3x^2\]
 \subsection{Nested brackets}
 example 1\\
 \[w=\left(\sqrt{1+\frac{1}{x^2}}\cdot\left[\log(2+\frac{1}{y})+ \left\{ 3^x - \left(\frac{2}{z}\right)^y\right\}\right]\right)\]

 example 2\\
 \[ z=\left\{\left(\frac{3}{2} + \left[ 5^2 -\left(4+\frac{1}{x}\right)^3\right]\right) \times(1+\sqrt{y})\right\} \]

\newpage
\section{Tables and Figures} 
happy synonyms\\

\begin{table}[h!]
\centering
\begin{tabular}{| c | c | c |}

\hline
Column 1 & Column 2 & Column 3\\ \hline
cheerful & delighted &ecstatic \\ \hline
glad & thrilled & jolly\\ \hline
jubilant & merry & upbeat\\ \hline
\end{tabular}

\caption{A simple table in LaTeX}
\label{Table 1:}
\end{table}

\begin{figure}[h!]
\centering
\includegraphics[width=0.5\textwidth]{iit dharwad}
\caption{IIT Dharwad}
\label{Figure 1:}
\end{figure}

\newpage
\section{Numbering}
\subsubsection{Unordered (Bullet) Lists}
\begin{itemize}
\item First item
\item Second item
\item Third item
\end{itemize}

\subsubsection{ordered (Numbered) Lists}
\begin{enumerate}
\item First item
\item Second item
\item Third item
\end{enumerate}

\subsubsection{Using Roman Numerals:}
\begin{enumerate}
\renewcommand{\labelenumi}{\Roman{enumi}.} 
\item First item
\item Second item
\item Third item
\end{enumerate}

\subsubsection{Using letters:}
\begin{enumerate}
\renewcommand{\labelenumi}{\alph{enumi})} 
\item First item
\item Second item
\item Third item
\end{enumerate}

\subsection{Nested Lists}
\begin{itemize}
\item First item
\begin{itemize}
\item Nested item 1
\item Nested item 2
\end{itemize}
\item Second item
\end{itemize}


\section{Pseudocode}
 Pseudocode is an informal way of programming description that does not re
quire any strict programming language syntax or underlying technology con
siderations. It is used for creating an outline or a rough draft of a program.
 Pseudocode summarizes a program’s flow, but excludes underlying details. La
TeX has several packages for typesetting algorithms in form of ”pseudocode”.
 They provide stylistic enhancements over a uniform style (i.e., all in typewriter
 font)
\newpage

\begin{algorithm}[H]
\caption{Merge Sort}
\begin{algorithmic}[1]
    \Procedure{MergeSort}{A, left, right}
    \If{left < right}
        \State mid $\gets$ floor((left + right)/2)
        \State \Call{MergeSort}{A, left, mid}
        \State \Call{MergeSort}{A, mid + 1, right}
        \State \Call{Merge}{A, left, mid, right}
    \EndIf
    \EndProcedure

    \Procedure{Merge}{A, left, mid, right}
    \State $n1 \gets$ mid $-$ left $+$ 1
    \State $n2 \gets$ right $-$ mid
    \State Create arrays $L[1 \dots n1]$ and $R[1 \dots n2]$
    \For{$i = 1$ to $n1$}
        \State $L[i] \gets A[left + i - 1]$
    \EndFor
    \For{$j = 1$ to $n2$}
        \State $R[j] \gets A[mid + j]$
    \EndFor
    \State $i, j, k \gets 1, 1, left$
    \While{$i \leq n1$ and $j \leq n2$}
        \If{$L[i] \leq R[j]$}
            \State $A[k] \gets L[i]$
            \State $i \gets i + 1$
        \Else
            \State $A[k] \gets R[j]$
            \State $j \gets j + 1$
        \EndIf
        \State $k \gets k + 1$
    \EndWhile
    \While{$i \leq n1$}
        \State $A[k] \gets L[i]$
        \State $i \gets i + 1$
        \State $k \gets k + 1$
    \EndWhile
    \While{$j \leq n2$}
        \State $A[k] \gets R[j]$
        \State $j \gets j + 1$
        \State $k \gets k + 1$
    \EndWhile
    \EndProcedure
\end{algorithmic}
\end{algorithm}

\newpage
\section{Coloring}
\textcolor{blue}{This is blue text.}\\
\textcolor{red}{This is blue text.}\\
\colorbox{yellow}{This text has a yellow background.}\\
\textcolor{purple}{This entire paragraph is purple. All sentences here will appear in purple. }

\section{Citing the papers}
\begin{itemize}
    \item Investigating the impact of network topology on the processing times of SDN controllers \cite{metter2015impact}
    \item SDN controllers: A comparative study \cite{salman2016comparative}
    \item Controllers in SDN: A Review Report \cite{paliwal2018controllers}
    \item Software Defined Networks: Comparative analysis of topologies with ONOS \cite{rajaratnam2017software}
\end{itemize}
\bibliographystyle{plain} % or any other style you prefer
\bibliography{ref} % The .bib filename without extension

\end{document}
